\documentclass[a4paper]{article}
\usepackage{ctex}
\usepackage{fontspec,xunicode,xltxtra}
\usepackage[pagestyles]{titlesec}
\usepackage{indentfirst}
\usepackage[top=1in,bottom=1in,left=1.25in,right=1.25in]{geometry}
\usepackage{amsmath}
\usepackage[usenames,dvipsnames]{xcolor}
\usepackage[colorlinks,linkcolor=red,anchorcolor=Blue,citecolor=green]{hyperref}

\setmainfont{Liberation Serif} 
\setsansfont{Liberation Sans} 
\setmonofont{WenQuanYi Micro Hei Mono}

\newcommand{\ftt}[1]{\fbox{\texttt{#1}}}
\punctstyle{kaiming}

\newcommand{\code}[1]{\mbox{\tt #1}}

\setlength{\parskip}{2pt}

\outer\def\beginlisp{%
  \begin{minipage}[t]{\linewidth}
  \begin{list}{$\bullet$}{%
    \setlength{\topsep}{0in}
    \setlength{\partopsep}{0in}
    \setlength{\itemsep}{0in}
    \setlength{\parsep}{0in}
    \setlength{\leftmargin}{1.5em}
    \setlength{\rightmargin}{0in}
    \setlength{\itemindent}{0in}
  }\item[]
  \obeyspaces
  \obeylines \footnotesize\tt}

\outer\def\endlisp{%
  \end{list}
  \end{minipage}
  }

{\obeyspaces\gdef {\ }}

\author{{信息科学技术学院~王迪~(学号:1300012802)}}

\begin{document}

\title{面向对象的程序设计语言}
\maketitle

\tableofcontents

\newpage

\section{Tutorial exercise 2}

\beginlisp
(define-class <vector> <object> xcor ycor)
\null
(define-method + ((v1 <vector>) (v2 <vector>))
  (make <vector> (xcor (+ (get-slot v1 'xcor)
                          (get-slot v2 'xcor)))
                 (ycor (+ (get-slot v1 'ycor)
                          (get-slot v2 'ycor)))))
\null
(define-method * ((v1 <vector>) (v2 <vector>))
  (+ (* (get-slot v1 'xcor) 
        (get-slot v2 'xcor)) 
     (* (get-slot v1 'ycor) 
        (get-slot v2 'ycor))))
\null
(define-method * ((v <vector>) (n <number>))
  (make <vector> (xcor (* (get-slot v 'xcor)
                          n))
                 (ycor (* (get-slot v 'ycor)
                          n))))
(define-method * ((n <number>) (v <vector>))
  (make <vector> (xcor (* n
                          (get-slot v 'xcor)))
                 (ycor (* n
                          (get-slot v 'ycor)))))
\null
(define-generic-function length)
(define-method length ((o <object>))
  (sqrt (* o o)))
\endlisp

\section{Tutorial exercise 3}

\code{paramlist-element-class}应该调用\code{tool-eval},因为类名不一定以常量符号即形如\code{<object>}给出,可能是一个合法表达式,需要对其求值。这样一来,我们在\code{define-method}时便获得了更大的灵活性。

\section{Tutorial exercise 4}

首先,解释器发现\code{say}是一个generic function,于是通过\code{generic-function-methods}获取了该function的所有methods,一共有3个。然后因为\code{fluffy}是\code{<house-cat>}而非\code{<show-cat>},所以会过滤掉1个,传给排序的methods其实只有2个:

\begin{enumerate}
    \item \code{say ((cat <cat>) (stuff <object>))}
    \item \code{say ((cat <cat>) (stuff <number>))}
\end{enumerate}

按照\code{method-more-specific?}谓词排序后,第2个method获得了较高的优先级,所以就调用了它。

\section{Tutorial exercise 5}

\beginlisp
(define-method print ((v <vector>))
  (print (cons 
           (get-slot v 'xcor)
           (get-slot v 'ycor))))
\endlisp

\section{Lab exercise 6}

在为\code{<vector>}定义\code{print}之前:

\beginlisp
TOOL==> (define v (make <vector> (xcor 1) (ycor 5)))
*undefined*
\null
TOOL==> v
(instance of <vector>)
\null
\endlisp

定义了\code{print}后:

\beginlisp
TOOL==> (define v (make <vector> (xcor 1) (ycor 5)))
*undefined*
\null
TOOL==> v
(1 . 5)
\endlisp

\section{Lab exercise 7}

我认为新的generic function应该限制在当前的eval环境中,而不是放进全局框架里。
\begin{itemize}
    \item 第一,从代码规范上来讲,如果一个generic function在全局范围内有作用,那么它应该显式地在全局进行定义,而不是在某个过程中被\code{define-method}隐式定义;
    \item 第二,从作用域上来讲,局部定义的generic function只在局部起作用,不仅合乎逻辑,也防止了局部的function名称污染全局环境;
    \item 第三,从效率上来讲,这样做提高了局部method寻找的效率,某种程度上也方便垃圾回收(一般来说,过程完成后,局部框架会回收,而因为加入的generic function与其他环境框架无关,所以也可以被回收)。
\end{itemize}

下面是一个例子:

\beginlisp
(define-method test ()
  (define-method method-in-test ((n <number>))
    (+ n 1)))
\null
(test)
(method-in-test 1)
\null
\endlisp

在我的修改版本中,最后一行调用会引发一个变量未约束的错误,而若是将generic function定义在了全局范围,最后一行调用则能成功,且返回值为2。

我在过程\code{eval-define-method}中添加了如下代码:

\beginlisp
(let ((var (method-definition-generic-function exp)))
  (if (variable? var)
    (let ((b (binding-in-env var env)))
      (if (or 
            (not (found-binding? b))
            (not (generic-function? (binding-value b))))
        (let ((val (make-generic-function var)))
          (define-variable! var val env))))))
\null
\endlisp

下面是一些测试:

\beginlisp
TOOL==> (define-method inc ((n <number>)) (+ n 1))
(added method to generic function: inc)
\null
TOOL==> (inc 5)
6
\null
(define-method inc ((l <list>)) (cons 1 l))
(added method to generic function: inc)
\null
TOOL==> (inc '(1 2 3))
(1 1 2 3)
\endlisp

\section{Lab exercise 8}

直接调用\code{tool-eval}实现,且基于了上一题的结果。在\code{eval-define-class}最后返回值前加入了如下代码:

\beginlisp
(for-each
  (lambda (slot-name)
    (tool-eval
      `(define-method ,slot-name ((obj ,name)) (get-slot obj ',slot-name))
      env))
  all-slots)
\null
\endlisp

代码第4行最左端是一个反引号。

下面是一些测试:

\beginlisp
TOOL==> (define-class <person> <object> name sex)
(defined class: <person>)
\null
TOOL==> (define me (make <person> (name 'wayne) (sex 'male)))
*undefined*
\null
TOOL==> (name me)
wayne
\null
TOOL==> (sex me)
male
\endlisp

\section{Lab exercise 9}

首先是一些关于\code{<vector>}的例子:

\beginlisp
TOOL==> (define-class <vector> <object> xcor ycor)
(defined class: <vector>)
\null
TOOL==> (define-method print ((v <vector>))
          (print (cons (xcor v) (ycor v))))
(added method to generic function: print)
\null
TOOL==> (define-method + ((v1 <vector>) (v2 <vector>))
          (make <vector>
                (xcor (+ (xcor v1) (xcor v2)))
                (ycor (+ (ycor v1) (ycor v2)))))
(added method to generic function: +)
\null
TOOL==> (define v1 
          (make <vector>
                (xcor (make <vector> (xcor 1) (ycor 5)))
                (ycor 4)))
*undefined*
TOOL==> (define v2
          (make <vector>
                (xcor (make <vector> (xcor -2) (ycor 2)))
                (ycor -1)))
*undefined*
\null
TOOL==> (+ v1 v2)
((instance (class <vector> ((class <object> () ())) (xcor ycor)) (-1 7)) . 3)
\null
TOOL==> (ycor v2)
-1
\null
TOOL==> (xcor v1)
(1 . 5)
\null
\endlisp

然后从\code{<vector>}类派生了\code{<3d-vector>}类:

\beginlisp
TOOL==> (define-class <3d-vector> <vector> zcor)
(defined class: <3d-vector>)
\null
TOOL==> (define v3
          (make <3d-vector>
                (xcor (make <vector> (xcor -1) (ycor 3)))
                (ycor 2)
                (zcor -3)))
*undefined*
\null
TOOL==> (zcor v3)
-3
\null
TOOL==> (xcor v3)
(-1 . 3)
\null
\endlisp

对generic function的调用进行了测试:

\beginlisp
\null
TOOL==> (+ v1 v3)
((instance (class <vector> ((class <object> () ())) (xcor ycor)) (0 8)) . 6)
\null
TOOL==> (define-method + ((v1 <vector>) (v2 <3d-vector>))
          (make <3d-vector>
                (xcor (+ (xcor v1) (xcor v2)))
                (ycor (+ (ycor v1) (ycor v2)))
                (zcor (+ (zcor v2) 100))))
(added method to generic function: +)
\null
TOOL==> (define-method print ((v <3d-vector>))
          (print (cons (xcor v) (cons (ycor v) (zcor v)))))
(added method to generic function: print)
\null
TOOL==> (+ v1 v3)
((instance (class <vector> ((class <object> () ())) (xcor ycor)) (0 8)) 6 . 97)
\null
\endlisp

可以看到7、8两个练习中的修改都工作得很好。

\end{document}
