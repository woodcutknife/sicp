\documentclass[a4paper]{article}
\usepackage{ctex}
\usepackage{fontspec,xunicode,xltxtra}
\usepackage[pagestyles]{titlesec}
\usepackage{indentfirst}
\usepackage[top=1in,bottom=1in,left=1.25in,right=1.25in]{geometry}
\usepackage{amsmath}
\usepackage[usenames,dvipsnames]{xcolor}
\usepackage[colorlinks,linkcolor=red,anchorcolor=Blue,citecolor=green]{hyperref}

\setmainfont{Liberation Serif} 
\setsansfont{Liberation Sans} 
\setmonofont{WenQuanYi Micro Hei Mono}

\newcommand{\ftt}[1]{\fbox{\texttt{#1}}}
\punctstyle{kaiming}

\newcommand{\Code}[1]{\mbox{\tt #1}}

\setlength{\parskip}{2pt}

\author{{信息科学技术学院~王迪~(学号:1300012802)}}

\begin{document}

\title{扩展通用算术包}
\maketitle

\section{项目内容}
按照Exercise完成了如下扩充:
\begin{enumerate}
    \item \textbf{5.1A 5.1B} 定义了通用的\Code{equ?}谓词,为Number包添加了\Code{=number}谓词并安装到算术包中。
    \item \textbf{5.3A 5.3B} 为Rational包添加了\Code{equ-rat?}谓词并安装到算术包中。
    \item \textbf{5.4A 5.4B} 定义\Code{repnum->reprat}过程,并以此实现Number与Rational间的四则运算和相等性判断。
    \item \textbf{5.5B} 定义\Code{create-numercial-polynomial}过程,通过一个变量符号和一个系数列表生成一个多项式。
    \item \textbf{5.5C} 补充了\Code{map-terms}的定义,并定义了通用的\Code{pretty-disp}方法,以更加友好的方式输出Number、Rational和Polynomial。
    \item \textbf{5.7A 5.7B 5.7C} 定义了\Code{negate-polynomial}过程,并以此实现了多项式的减法和相等性判断,且安装到算术包中。
    \item \textbf{5.8A 5.8B} 定义\Code{repnum->reppoly}过程,并以此实现Number和Polynomial间的四则运算和相等性判断。
    \item \textbf{5.9A 5.9B} 补充了\Code{apply-terms}的定义,使得通用多项式可以进行代入求值。
\end{enumerate}

\section{基本设计}
顺应了提供代码的思路,可以看出核心思想就是数据导向的程序设计。

通过良好设计的抽象屏障,不用关心通用算术包的底层数据结构,通过查看提供的接口就可以方便地进行功能扩充。另外,使用表格对通用数字的各种运算进行管理,结构简单、清晰。

\section{所遇问题}
因为提供的代码结构良好,所以在扩充功能时没有碰到什么大问题。

有一个小问题:为了进行Number和Rational之间的运算,定义了\Code{NNmethod->NRmethod}过程将一个$(\mbox{RepNum},\mbox{RepNum}) \to T$的过程变为$(\mbox{RepNum},\mbox{RepRat}) \to T$的过程。但这样只能处理\Code{(number rational)}的参数类型,无法直接处理\Code{(rational number)}的参数类型。

解决方法是写了一个类似的\Code{NNmethod->RNmethod}过程,这么做是最简单的解决方法,但一方面这出现了重复代码,要修改比较繁琐,另一方面也不易扩展到更多参数时的类型转换。

\section{改进方法}
我认为这个系统最大的缺陷在于进行类型转换很繁琐。一个改进方法就是课本上所说的,为每个类型添加从它到其他类型的转换过程,然后修改\Code{apply-generic}的定义,较为智能地去完成类型转换。

\end{document}
